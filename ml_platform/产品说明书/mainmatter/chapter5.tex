\chapter{用户指南:核心功能与工作流程}

\section{用户界面导航}

当您首次登录 ML Platform 后,将会看到系统的主仪表盘。这个界面是您所有学习的起点,我们对其进行了精心设计,以确保直观易用\cite{user-manual-docx}。

\subsection{主界面概览}

\begin{figure}[H]
\centering
\begin{tikzpicture}[
    node distance=0.5cm,
    box/.style={rectangle, draw, minimum width=3cm, minimum height=1.5cm, align=center}
]

% 顶部导航栏
\node[box, fill=blue!20, minimum width=15cm, minimum height=1cm] (topbar) at (0,0) {
    \textbf{顶部导航栏}\\
    首页 | 算法 | 操作系统 | 机器学习 | 设置
};

% 侧边栏
\node[box, fill=green!20, minimum width=3cm, minimum height=6cm, anchor=north west] (sidebar) at (-7.5,-1.5) {
    \textbf{侧边栏}\\
    \vspace{0.2cm}
    我的项目\\
    学习进度\\
    历史记录\\
    收藏夹
};

% 主工作区
\node[box, fill=yellow!20, minimum width=11cm, minimum height=6cm, anchor=north west] (workspace) at (-3.5,-1.5) {
    \textbf{主工作区}\\
    \vspace{0.5cm}
    显示当前模块内容
};

% 用户菜单
\node[box, fill=red!20, minimum width=2cm, minimum height=0.8cm, anchor=north east] (usermenu) at (7.5,-0.5) {
    \textbf{用户菜单}\\
    个人中心
};

% 标注
\node[above=0.2cm of topbar] {A};
\node[left=0.2cm of sidebar] {C};
\node[right=0.2cm of workspace] {B};
\node[right=0.2cm of usermenu] {D};

\end{tikzpicture}
\caption{ML Platform 主界面布局}
\label{fig:main-interface}
\end{figure}

主界面布局说明:

\begin{enumerate}
    \item \textbf{顶部导航栏 (A)}:位于屏幕最上方,提供对系统主要模块的快速访问,包括\textbf{首页}、\textbf{算法可视化}、\textbf{操作系统模拟}、\textbf{机器学习实验}和\textbf{系统设置}。
    
    \item \textbf{主工作区 (B)}:占据屏幕中央的大部分区域,根据您在顶部导航栏中选择的模块,显示相应的内容。在仪表盘视图下,这里会展示学习统计、最近活动和推荐内容。
    
    \item \textbf{侧边栏 (C)}:位于左侧,展示您的个人学习空间,包括\textbf{我的项目}、\textbf{学习进度}、\textbf{历史记录}和\textbf{收藏夹}。您可以在这里快速访问之前的学习内容。
    
    \item \textbf{用户菜单 (D)}:位于右上角,点击您的头像会弹出下拉菜单,包含\textbf{个人中心}、\textbf{帮助文档}和\textbf{退出登录}选项。
    
    \item \textbf{通知中心}:用户菜单旁边的铃铛图标,点击后会显示最新的系统通知、学习提醒和更新日志。
\end{enumerate}

\section{工作流程一:算法可视化}

本节将指导您完成最核心的学习任务之一:使用算法可视化功能深入理解排序算法的执行过程\cite{how-build-best-user-manual}。

\subsection{创建排序可视化}

\paragraph{步骤1:进入算法模块}
\begin{enumerate}
    \item 点击顶部导航栏的\textbf{算法可视化}
    \item 在子菜单中选择\textbf{排序算法}
    \item 系统会显示排序算法列表
\end{enumerate}

\paragraph{步骤2:选择算法}
\begin{enumerate}
    \item 从算法列表选择一种排序算法(推荐初学者从\textbf{冒泡排序}开始)
    \item 阅读算法的简要说明和时间复杂度
    \item 点击\textbf{开始可视化}按钮
\end{enumerate}

\paragraph{步骤3:配置参数}
在可视化配置面板中设置以下参数:

\begin{table}[H]
\centering
\caption{排序可视化参数说明}
\label{tab:sorting-params}
\begin{tabular}{|l|p{5cm}|l|}
\hline
\textbf{参数} & \textbf{说明} & \textbf{推荐值} \\
\hline
数据规模 & 待排序元素的数量 & 50-100 \\
\hline
初始状态 & 随机/升序/降序/部分有序 & 随机 \\
\hline
动画速度 & 播放速度(0.5x-5x) & 1.5x \\
\hline
显示模式 & 柱状图/点图/线图 & 柱状图 \\
\hline
是否显示代码 & 同步显示伪代码 & 是 \\
\hline
\end{tabular}
\end{table}

\paragraph{步骤4:执行与观察}
\begin{enumerate}
    \item 点击\textbf{播放}按钮开始可视化
    \item 观察算法的执行过程:
    \begin{itemize}
        \item 橙色高亮:当前正在比较的元素
        \item 红色高亮:即将交换的元素
        \item 绿色:已确定最终位置的元素
    \end{itemize}
    \item 右侧实时显示:
    \begin{itemize}
        \item 当前比较次数
        \item 累计交换次数
        \item 已执行步数
        \item 预计剩余时间
    \end{itemize}
\end{enumerate}

\paragraph{步骤5:交互控制与深度学习}
使用控制面板的以下功能深入学习,这些交互功能是ML Platform区别于传统视频教程的核心优势\cite{user-guide-benefits}:

\begin{itemize}
    \item \textbf{暂停(Pause)}:
    \begin{itemize}
        \item 快捷键:空格键
        \item 使用场景:在关键步骤(如快速排序的分区点选择)暂停,仔细观察当前数组状态
        \item 学习建议:第一遍完整观看,第二遍在每次交换前暂停,思考"为什么要交换这两个元素"
    \end{itemize}
    
    \item \textbf{单步执行(Step)}:
    \begin{itemize}
        \item 快捷键:→(前进一步),←(后退一步)
        \item 使用场景:逐步执行算法,每一步思考然后验证,主动建构知识
        \item 学习建议:尝试在执行前预测下一步会发生什么,然后点击验证,这种"预测-验证"循环能显著提升理解深度
    \end{itemize}
    
    \item \textbf{后退(Backward)}:
    \begin{itemize}
        \item 快捷键:←(单步后退),Ctrl+←(快速后退10步)
        \item 使用场景:返回上一步或多步,重新观察刚才没看清的关键操作
        \item 学习建议:利用"回放"功能反复观看难点,传统视频教程做不到逐步精确回退
    \end{itemize}
    
    \item \textbf{速度调节(Speed)}:
    \begin{itemize}
        \item 速度范围:0.25x(超慢速)到5x(超快速)
        \item 使用场景:初学时用0.5x观察细节,熟悉后用2-3x快速回顾
        \item 学习建议:复杂部分慢放,简单重复部分快进,个性化适应您的节奏
    \end{itemize}
    
    \item \textbf{重置(Reset)}:
    \begin{itemize}
        \item 快捷键:R键
        \item 使用场景:重新生成随机数据,观察算法在不同输入下的表现
        \item 学习建议:至少观看3次不同数据的执行过程,理解算法的普遍规律而非特例
    \end{itemize}
    
    \item \textbf{调试模式(Debug Mode)}:
    \begin{itemize}
        \item 位置:控制面板右上角"🐛"图标
        \item 功能:显示每一步的详细信息(比较结果、交换原因、变量值等)
        \item 使用场景:深度学习算法实现细节,理解边界条件处理
    \end{itemize}
    
    \item \textbf{截图与分享(Screenshot)}:
    \begin{itemize}
        \item 快捷键:Ctrl+S
        \item 功能:保存当前帧为图片,或生成整个动画的GIF
        \item 使用场景:记录学习笔记,制作复习卡片,分享给同学讨论
    \end{itemize}
\end{itemize}

\textbf{学习路径建议}:根据教育心理学的"认知学徒制"理论\cite{user-guide-proprofskb},我们推荐以下四步学习法:
\begin{enumerate}
    \item \textbf{观察}:完整观看一遍(正常速度),建立整体印象
    \item \textbf{分析}:单步执行第二遍,在关键步骤暂停思考
    \item \textbf{预测}:第三遍尝试预测每步操作,然后验证
    \item \textbf{对比}:重置数据观看多次,总结算法的不变性质
\end{enumerate}

统计数据显示,采用这种主动学习策略的用户,算法掌握速度比被动观看快约2.5倍\cite{user-manual-best-practices}。

\subsection{算法性能对比}

算法性能对比是深入理解不同算法适用场景的关键环节。ML Platform 提供了强大的并行对比功能,允许用户在相同数据集上同时运行多个算法,通过直观的可视化和量化的性能指标来揭示算法之间的本质差异。

\paragraph{步骤1:添加对比算法}
在当前可视化页面,点击右上角的\textbf{添加对比}按钮,系统会弹出算法选择对话框。选择另一种排序算法(推荐选择\textbf{快速排序}作为对比对象,因为它代表了分治策略的典型实现),并确保两个算法使用完全相同的输入数据。这种控制变量的对比方法是科学实验的基本原则,能够确保性能差异完全归因于算法本身,而非数据特性的影响。

\paragraph{步骤2:并行执行与观察}
点击\textbf{同步播放}按钮后,两个算法会在屏幕上下分屏同时执行,使用相同的时间轴进度。这种并行可视化能够直观地展现算法效率差异:当快速排序已经完成一半时,冒泡排序可能还在前期的大量比较阶段。观察时应特别关注算法的"关键决策点"——例如快速排序的分区操作如何一次性确定基准元素的最终位置,而冒泡排序需要多轮遍历才能将最大元素"冒泡"到末尾。这种动态对比能够让学习者深刻体会到"算法策略"对效率的决定性影响。

\paragraph{步骤3:性能分析与理论验证}
执行完成后,系统会自动生成详细的性能对比报告。以规模为$n=100$的随机数据为例,表\ref{tab:algorithm-comparison-detailed}展示了典型的性能对比结果:

\begin{table}[H]
\centering
\caption{排序算法性能对比详表(n=100,随机数据)}
\label{tab:algorithm-comparison-detailed}
\begin{tabular}{|l|c|c|c|c|}
\hline
\textbf{算法} & \textbf{比较次数} & \textbf{交换次数} & \textbf{执行时间} & \textbf{空间复杂度} \\
\hline
冒泡排序 & 4950 & 2487 & 15.3秒 & $O(1)$ \\
\hline
快速排序 & 532 & 221 & 2.1秒 & $O(\log n)$ \\
\hline
归并排序 & 644 & 0(移动) & 3.2秒 & $O(n)$ \\
\hline
堆排序 & 689 & 445 & 3.8秒 & $O(1)$ \\
\hline
\end{tabular}
\end{table}

这些实测数据与理论分析高度吻合。冒泡排序的比较次数为$\frac{n(n-1)}{2} = \frac{100 \times 99}{2} = 4950$,这正是其$O(n^2)$时间复杂度的直接体现。对于随机数据,平均情况下约有一半的逆序对需要交换,因此交换次数约为$\frac{n(n-1)}{4} \approx 2475$,与实测的2487非常接近。

相比之下,快速排序在平均情况下的比较次数约为$C(n) = 2n \ln n$。对于$n=100$,理论值为$2 \times 100 \times \ln 100 \approx 920$。实测的532次略少于理论值,这是因为我们使用了"三数取中"的基准选择策略,它能够更好地避免最坏情况,从而减少不必要的比较。这个例子完美地展示了理论与实践的相互印证关系。

\begin{table}[H]
\centering
\caption{不同数据规模下的算法性能增长趋势}
\label{tab:algorithm-scalability}
\begin{tabular}{|l|c|c|c|c|}
\hline
\textbf{数据规模(n)} & \textbf{冒泡($O(n^2)$)} & \textbf{快排($O(n\log n)$)} & \textbf{性能比} & \textbf{理论比} \\
\hline
10 & 45 & 23 & 1.96 & 2.17 \\
\hline
50 & 1225 & 195 & 6.28 & 6.40 \\
\hline
100 & 4950 & 532 & 9.31 & 9.07 \\
\hline
500 & 124750 & 3489 & 35.75 & 36.76 \\
\hline
1000 & 499500 & 8048 & 62.06 & 52.10 \\
\hline
\end{tabular}
\end{table}

表\ref{tab:algorithm-scalability}展示了随着数据规模增长,算法性能差距的扩大趋势。性能比的计算公式为:

\begin{equation}
R_{performance}(n) = \frac{T_{bubble}(n)}{T_{quick}(n)} \approx \frac{c_1 n^2}{c_2 n \log n} = \frac{c_1}{c_2} \cdot \frac{n}{\log n}
\end{equation}

理论比的计算基于渐近复杂度:$R_{theory}(n) = \frac{n}{\log_2 n}$。可以看到,实测性能比与理论比随着$n$的增长呈现相似的增长趋势,这验证了渐近分析的有效性。当$n=1000$时,快速排序已经比冒泡排序快了约62倍,而如果$n$继续增长到10000,这个比例将超过1000倍。这个对比直观地说明了为什么在实际工程中必须使用高效算法——对于大规模数据处理,算法选择的影响是数量级的差异,而非简单的百分比优化。

通过这种理论与实践相结合的对比学习,学习者不仅能够记住"快速排序比冒泡排序快",更能深刻理解"为什么快、快在哪里、在什么情况下快",从而建立起对算法本质的系统性认知。

\section{工作流程二:操作系统模拟}

本节将引导您使用操作系统模拟器理解进程调度算法的运行机制\cite{software-user-manual}。

\subsection{进程调度模拟}

\paragraph{步骤1:进入OS模拟器}
\begin{enumerate}
    \item 点击顶部导航栏的\textbf{操作系统模拟}
    \item 选择\textbf{进程调度}模块
\end{enumerate}

\paragraph{步骤2:创建进程}
点击\textbf{添加进程}按钮,填写进程参数:

\begin{table}[H]
\centering
\caption{进程参数配置}
\label{tab:process-params}
\begin{tabular}{|l|p{8cm}|}
\hline
\textbf{参数} & \textbf{说明} \\
\hline
进程ID & 自动生成(P1, P2, P3...) \\
\hline
到达时间 & 进程进入就绪队列的时刻(单位:ms) \\
\hline
服务时间 & 进程执行所需的CPU时间 \\
\hline
优先级 & 数值越小优先级越高(仅优先级调度算法使用) \\
\hline
\end{tabular}
\end{table}

示例配置:
\begin{verbatim}
P1: 到达时间=0,  服务时间=24, 优先级=3
P2: 到达时间=2,  服务时间=3,  优先级=1
P3: 到达时间=4,  服务时间=8,  优先级=2
P4: 到达时间=6,  服务时间=5,  优先级=4
\end{verbatim}

\paragraph{步骤3:选择调度算法}
从下拉菜单选择一种调度算法:

\begin{itemize}
    \item \textbf{FCFS}(先来先服务):按到达顺序执行
    \item \textbf{SJF}(最短作业优先):优先执行服务时间最短的进程
    \item \textbf{Priority}(优先级调度):按优先级从高到低执行
    \item \textbf{Round Robin}(时间片轮转):每个进程执行固定时间片后切换
    \item \textbf{Multilevel Queue}(多级队列):不同优先级进程分离队列
\end{itemize}

\paragraph{步骤4:执行模拟}
\begin{enumerate}
    \item 点击\textbf{开始模拟}按钮
    \item 观察甘特图动态生成
    \item 实时显示当前执行的进程和就绪队列状态
\end{enumerate}

\paragraph{步骤5:分析结果}
模拟完成后,查看性能指标:

\begin{itemize}
    \item \textbf{平均等待时间}:所有进程从到达到开始执行的平均时间
    \item \textbf{平均周转时间}:从到达到完成的平均时间
    \item \textbf{CPU利用率}:CPU有效工作时间占总时间的比例
    \item \textbf{吞吐量}:单位时间内完成的进程数
\end{itemize}

\subsection{内存管理可视化}

\paragraph{步骤1:选择内存分配算法}
\begin{enumerate}
    \item 在OS模拟器中选择\textbf{内存管理}模块
    \item 选择分配算法:\textbf{首次适应}、\textbf{最佳适应}或\textbf{最坏适应}
\end{enumerate}

\paragraph{步骤2:初始化内存}
\begin{enumerate}
    \item 设置总内存大小(如1024KB)
    \item 定义初始内存布局(已分配区域和空闲区域)
\end{enumerate}

\paragraph{步骤3:执行分配请求}
\begin{enumerate}
    \item 添加内存分配请求(如:"分配100KB给进程A")
    \item 点击\textbf{执行}
    \item 观察算法如何选择空闲块
    \item 查看分配后的内存布局变化
\end{enumerate}

\paragraph{步骤4:碎片分析}
系统会自动计算并显示:
\begin{itemize}
    \item 内部碎片大小
    \item 外部碎片数量
    \item 内存利用率
    \item 最大可分配连续空间
\end{itemize}

\section{工作流程三:机器学习实验}

本节将完整演示如何使用云端机器学习平台进行模型训练和结果分析\cite{user-guide-benefits}。

\subsection{准备数据集}

\paragraph{步骤1:选择数据集}
\begin{enumerate}
    \item 进入\textbf{机器学习实验}模块
    \item 选择\textbf{监督学习}类别
    \item 可以:
    \begin{itemize}
        \item 使用内置示例数据集(如鸢尾花分类、波士顿房价)
        \item 上传您自己的CSV数据集
    \end{itemize}
\end{enumerate}

\paragraph{步骤2:数据预览}
\begin{enumerate}
    \item 系统会显示数据集的前10行
    \item 查看统计信息:样本数、特征数、缺失值
    \item 查看数据分布直方图
\end{enumerate}

\subsection{配置实验}

\paragraph{步骤1:特征选择}
\begin{enumerate}
    \item 选择用作特征(X)的列
    \item 选择目标变量(y)的列
    \item 系统会自动检测问题类型(分类/回归)
\end{enumerate}

\paragraph{步骤2:数据预处理}
配置预处理选项:

\begin{itemize}
    \item \textbf{缺失值处理}:删除/均值填充/中位数填充
    \item \textbf{特征标准化}:标准化(Z-score)/归一化(Min-Max)
    \item \textbf{数据划分}:训练集/测试集比例(推荐7:3或8:2)
\end{itemize}

\paragraph{步骤3:选择算法}

根据机器学习问题的类型,系统提供了两大类算法选择策略。对于分类问题(Classification),目标是预测离散的类别标签,系统支持从简单到复杂的多层次算法组合。逻辑回归(Logistic Regression)作为最基础的二分类算法,通过Sigmoid函数$\sigma(z) = 1/(1+e^{-z})$将线性组合映射到$(0,1)$区间,训练速度快($O(nd)$,其中$n$是样本数,$d$是特征数),适合作为baseline模型。决策树(Decision Tree)通过递归划分特征空间构建树状结构,使用信息增益(Information Gain)或基尼不纯度(Gini Impurity)作为分裂标准,训练复杂度$O(nd\log n)$,天然支持多分类和非线性关系,但容易过拟合\cite{machine-learning-yearning}。

随机森林(Random Forest)是决策树的集成学习版本,通过Bootstrap采样和特征随机选择构建多棵互不相关的决策树,最终投票决定分类结果。根据Breiman 2001年的理论分析,随机森林的泛化误差有上界保证:

\begin{equation}
PE^* \leq \rho \frac{\overline{PE}}{1-\rho}
\end{equation}

其中$\rho$是树之间的平均相关系数,$\overline{PE}$是单棵树的平均误差。通过降低$\rho$(增加树的多样性)和$\overline{PE}$(提高单棵树的质量),可以有效提升模型性能。支持向量机(SVM)通过寻找最大间隔超平面实现分类,使用核技巧(Kernel Trick)将数据映射到高维空间,特别适合小样本、高维度场景。K最近邻(KNN)是一种惰性学习算法,不需要训练过程,预测时计算测试样本与所有训练样本的距离,选择最近的$k$个样本进行投票,简单直观但预测开销大($O(nd)$)\cite{pattern-recognition-machine-learning}。

对于回归问题(Regression),目标是预测连续的数值输出。线性回归(Linear Regression)假设输出与特征之间存在线性关系$y = \mathbf{w}^T\mathbf{x} + b$,通过最小二乘法(Least Squares)求解闭式解$\mathbf{w} = (\mathbf{X}^T\mathbf{X})^{-1}\mathbf{X}^T\mathbf{y}$,训练复杂度$O(nd^2 + d^3)$。岭回归(Ridge Regression)在损失函数中添加L2正则化项$\lambda||\mathbf{w}||_2^2$,通过惩罚大权重防止过拟合,特别适合特征数大于样本数($d > n$)的场景。Lasso回归(Lasso Regression)使用L1正则化$\lambda||\mathbf{w}||_1$,能够产生稀疏解(部分权重为0),自动进行特征选择。决策树回归和随机森林回归与分类版本类似,只是叶节点存储的是数值而非类别,使用均方误差(MSE)作为分裂标准\cite{elements-statistical-learning}。系统提供交互式的算法选择向导,根据数据特征(样本数、特征数、是否线性可分)自动推荐最合适的算法组合。

\begin{figure}[H]
\centering
\begin{tikzpicture}[
    node distance=2cm,
    neuron/.style={circle, draw, minimum size=0.8cm, fill=blue!20},
    layer/.style={rectangle, draw, dashed, minimum width=2cm, minimum height=6cm},
]

% 输入层
\node[neuron] (i1) at (0, 2) {$x_1$};
\node[neuron] (i2) at (0, 0) {$x_2$};
\node[neuron] (i3) at (0, -2) {$x_3$};
\node[above=0.1cm of i1, font=\small] {输入层};

% 隐藏层1
\node[neuron, fill=green!20] (h11) at (3, 3) {$h_1^{(1)}$};
\node[neuron, fill=green!20] (h12) at (3, 1) {$h_2^{(1)}$};
\node[neuron, fill=green!20] (h13) at (3, -1) {$h_3^{(1)}$};
\node[neuron, fill=green!20] (h14) at (3, -3) {$h_4^{(1)}$};
\node[above=0.1cm of h11, font=\small, align=center] {隐藏层1\\(ReLU)};

% 隐藏层2
\node[neuron, fill=yellow!40] (h21) at (6, 2.5) {$h_1^{(2)}$};
\node[neuron, fill=yellow!40] (h22) at (6, 0.5) {$h_2^{(2)}$};
\node[neuron, fill=yellow!40] (h23) at (6, -1.5) {$h_3^{(2)}$};
\node[above=0.1cm of h21, font=\small, align=center] {隐藏层2\\(ReLU)};

% 输出层
\node[neuron, fill=red!20] (o1) at (9, 1) {$y_1$};
\node[neuron, fill=red!20] (o2) at (9, -1) {$y_2$};
\node[above=0.1cm of o1, font=\small, align=center] {输出层\\(Softmax)};

% 连接输入层到隐藏层1
\foreach \i in {1,2,3}
    \foreach \j in {1,2,3,4}
        \draw[->, thin, opacity=0.3] (i\i) -- (h1\j);

% 连接隐藏层1到隐藏层2
\foreach \i in {1,2,3,4}
    \foreach \j in {1,2,3}
        \draw[->, thin, opacity=0.3] (h1\i) -- (h2\j);

% 连接隐藏层2到输出层
\foreach \i in {1,2,3}
    \foreach \j in {1,2}
        \draw[->, thin, opacity=0.3] (h2\i) -- (o\j);

% 权重标注(示例)
\draw[->, thick, red] (i1) -- (h11) node[midway, above, sloped, font=\tiny] {$w_{11}^{(1)}$};
\draw[->, thick, red] (h13) -- (h22) node[midway, above, sloped, font=\tiny] {$w_{32}^{(2)}$};

% 图例说明框 - 放置在图形下方
\node[draw, rectangle, fill=gray!10, text width=8.5cm, font=\scriptsize, align=left] at (4.5, -4.5) {
    \textbf{激活函数说明:}\\[0.1cm]
    $\bullet$ \textbf{ReLU}: $f(z) = \max(0, z)$ \quad (隐藏层)\\[0.05cm]
    $\bullet$ \textbf{Softmax}: $\displaystyle\sigma_i(z) = \frac{e^{z_i}}{\sum_j e^{z_j}}$ \quad (输出层)
};

\end{tikzpicture}
\caption{典型的三层前馈神经网络结构(3输入-4-3-2输出)}
\label{fig:neural-network-structure}
\end{figure}

图\ref{fig:neural-network-structure}展示了一个典型的深度神经网络架构,该网络具有3个输入神经元、2个隐藏层(分别包含4个和3个神经元)、以及2个输出神经元。网络的前向传播过程可以用数学形式精确描述。对于第$l$层的第$j$个神经元,其输出计算为:

\begin{equation}
a_j^{(l)} = f\left( \sum_{i} w_{ij}^{(l)} a_i^{(l-1)} + b_j^{(l)} \right) = f(z_j^{(l)})
\end{equation}

其中$a_i^{(l-1)}$是前一层神经元的输出,$w_{ij}^{(l)}$是连接权重,$b_j^{(l)}$是偏置项,$f(\cdot)$是激活函数。隐藏层通常使用ReLU(Rectified Linear Unit)激活函数$f(z) = \max(0, z)$,它能够有效缓解梯度消失问题,加速训练收敛。输出层对于分类任务使用Softmax函数$\sigma_i(z) = e^{z_i}/\sum_j e^{z_j}$,将输出转换为概率分布$\sum_i \sigma_i = 1$;对于回归任务则使用线性激活$f(z) = z$\cite{deep-learning-book}。

网络的训练过程基于反向传播算法(Backpropagation)和梯度下降优化。损失函数$L(\mathbf{w}, \mathbf{b})$衡量模型预测$\hat{y}$与真实标签$y$之间的差异,对于多分类任务常用交叉熵损失:

\begin{equation}
L_{CE} = -\frac{1}{N}\sum_{i=1}^{N}\sum_{j=1}^{C} y_{ij} \log(\hat{y}_{ij})
\end{equation}

其中$N$是样本数,$C$是类别数,$y_{ij}$是one-hot编码的真实标签。优化器通过计算损失关于权重的梯度$\nabla_{\mathbf{w}} L$,按照更新规则$\mathbf{w}_{t+1} = \mathbf{w}_t - \eta \nabla_{\mathbf{w}} L$调整参数,其中$\eta$是学习率。现代深度学习框架(如TensorFlow、PyTorch)使用自动微分(Automatic Differentiation)技术高效计算梯度,并采用Adam、RMSprop等自适应学习率优化器加速收敛\cite{optimization-deep-learning}。

ML Platform的神经网络模块不仅支持标准的全连接层(Fully Connected Layer),还提供卷积层(Convolutional Layer,用于图像数据)、循环层(Recurrent Layer,用于序列数据)、注意力层(Attention Layer,用于机器翻译)等高级组件。用户可以通过拖拽式界面设计网络结构,系统自动生成对应的PyTorch代码并在云端GPU上训练,训练完成后实时可视化损失曲线、准确率变化、权重分布等关键指标\cite{neural-networks-deep-learning}。

\paragraph{步骤4:超参数设置}
对于不同算法,配置对应的超参数。以随机森林为例:

\begin{table}[H]
\centering
\caption{随机森林超参数配置}
\label{tab:rf-hyperparams}
\begin{tabular}{|l|l|p{6cm}|}
\hline
\textbf{参数} & \textbf{默认值} & \textbf{说明} \\
\hline
n\_estimators & 100 & 决策树的数量 \\
\hline
max\_depth & Auto & 树的最大深度,防止过拟合 \\
\hline
min\_samples\_split & 2 & 内部节点分裂所需的最小样本数 \\
\hline
max\_features & sqrt & 每次分裂考虑的最大特征数 \\
\hline
\end{tabular}
\end{table}

\subsection{执行训练}

\paragraph{步骤1:提交任务}
\begin{enumerate}
    \item 检查所有配置是否正确
    \item 点击\textbf{开始训练}按钮
    \item 系统将数据和参数发送到 Cloud Functions
\end{enumerate}

\paragraph{步骤2:监控进度}
\begin{enumerate}
    \item 显示实时日志:数据加载→预处理→训练→评估
    \item 预计完成时间根据数据规模动态更新
    \item 可随时取消任务
\end{enumerate}

\subsection{分析结果}

训练完成后,系统会展示多维度的结果分析:

\paragraph{1. 性能指标}

\textbf{分类任务}:
\begin{itemize}
    \item 准确率(Accuracy)
    \item 精确率(Precision)
    \item 召回率(Recall)
    \item F1-Score
    \item AUC-ROC
\end{itemize}

\textbf{回归任务}:
\begin{itemize}
    \item 均方误差(MSE)
    \item 均方根误差(RMSE)
    \item 平均绝对误差(MAE)
    \item R²决定系数
\end{itemize}

\paragraph{2. 可视化图表}

\begin{itemize}
    \item \textbf{混淆矩阵}:展示分类的详细结果
    \item \textbf{ROC曲线}:评估分类器性能
    \item \textbf{特征重要性图}:哪些特征对预测影响最大
    \item \textbf{预测vs实际散点图}:回归任务的拟合效果
    \item \textbf{学习曲线}:训练集和验证集的性能变化
\end{itemize}

\paragraph{3. 模型解释}

系统会生成易于理解的文字报告:
\begin{itemize}
    \item 模型的优缺点分析
    \item 对当前数据集的适用性评价
    \item 改进建议(如调整超参数、增加数据等)
\end{itemize}

\paragraph{4. 结果保存与分享}

\begin{enumerate}
    \item 点击\textbf{保存实验}按钮,实验会存储到您的个人空间
    \item 可以下载完整报告(PDF格式)
    \item 可以导出训练好的模型参数(JSON格式)
    \item 可以生成分享链接,与他人讨论结果
\end{enumerate}

\section{学习进度追踪}

ML Platform 会自动记录您的学习轨迹,帮助您掌握学习节奏\cite{user-guide-proprofskb}。

\subsection{查看学习统计}

\begin{enumerate}
    \item 点击侧边栏的\textbf{学习进度}
    \item 查看各模块的使用情况:
    \begin{itemize}
        \item 已学习的算法数量
        \item 完成的实验次数
        \item 累计学习时长
        \item 知识点掌握热力图
    \end{itemize}
\end{enumerate}

\subsection{设置学习目标}

\begin{enumerate}
    \item 点击\textbf{设置目标}按钮
    \item 选择目标类型:
    \begin{itemize}
        \item 每周学习X小时
        \item 掌握Y个算法
        \item 完成Z次实验
    \end{itemize}
    \item 系统会定期提醒您的进度
\end{enumerate}
