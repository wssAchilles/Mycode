\begin{abstract}

计算机科学教育中普遍存在算法原理抽象、操作系统机制难以理解、机器学习理论与实践脱节等问题,传统教学方法缺乏交互性,严重制约学习效率。针对408考研等专业课程学习需求,亟需一种融合可视化、交互式实验与智能分析的综合性教学平台。ML Platform 机器学习可视化实验平台正是为解决上述问题而设计开发的系统化解决方案,旨在通过可视化技术和云端计算能力,为计算机专业学生、考研学习者及教育工作者提供高效的理论学习与实践验证工具。

本平台采用 Flutter 跨平台框架构建前端交互界面,基于 Firebase 云服务实现数据存储与用户认证,使用 Python 机器学习库提供后端计算支持。系统集成三大核心模块:算法可视化动画引擎支持排序、查找、图论等40余种经典算法的动态演示与步进调试;操作系统模拟器实现进程调度、内存管理、文件系统等机制的交互式模拟;机器学习实验平台提供20余种常用算法的在线训练、评估与可视化分析。系统架构采用微服务设计,支持 Web、Windows、Android、iOS、macOS 等多平台部署\cite{flutter2023},确保学习者可在不同终端设备上无缝使用。

ML Platform 实现了算法执行过程的实时可视化追踪,数据结构变化的动画呈现,以及机器学习模型的交互式调参与效果对比。系统支持用户自定义算法上传、批量实验管理、RESTful API 集成等高级功能,提供完整的权限管理与数据安全保障机制。本说明书遵循 IEEE 标准文档规范编写,包含详尽的安装配置指南、功能使用说明、API 参考文档及故障排除方案,面向计算机专业学生、考研学习者、教师及开发人员,系统阐述平台的技术架构、功能特性、使用方法及维护策略,为读者全面掌握和高效使用 ML Platform 提供权威指导。
	\thusetup{
		keywords = {算法可视化, 机器学习, 云端实验平台, 计算机教育, 跨平台应用},
	}
\end{abstract}

\begin{abstract*}
	Traditional computer science education struggles with abstract algorithm concepts, operating system mechanisms, and the theory-practice gap in machine learning. ML Platform addresses these challenges through an integrated visualization and experimentation platform designed for computer science students and educators.
	
	Built on Flutter and Firebase, the platform integrates three core modules: (1) an algorithm visualization engine supporting 40+ classic algorithms with step-by-step animation, (2) an operating system simulator for process scheduling and memory management, and (3) a machine learning experiment platform with 20+ algorithms leveraging cloud-based Python computation. The microservice architecture enables cross-platform deployment across Web, Windows, Android, iOS, and macOS\cite{flutter2023}.
	
	Key features include real-time execution tracking, interactive parameter tuning, custom algorithm upload, and comprehensive performance analysis. This manual provides detailed technical documentation following IEEE standards, covering system architecture, implementation guidelines, API references, and operational procedures for effective platform utilization.
	
	\thusetup{
		keywords* = {Algorithm Visualization, Machine Learning, Cloud Experiment Platform, Computer Science Education, Cross-platform Application},
	}
\end{abstract*}