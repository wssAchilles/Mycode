\chapter{界面与非功能性需求}

\section{用户界面 (UI/UX) 设计}
\subsection{设计理念}
\begin{itemize}
    \item \textbf{简洁直观:} 界面设计避免不必要的装饰,聚焦于内容本身,让用户可以专注于学习和交互。
    \item \textbf{引导性强:} 通过清晰的布局和视觉提示,引导用户进行操作和探索,降低使用门槛。
    \item \textbf{一致性:} 保持Web端和Android端在布局、色彩和交互方式上的高度一致性,提供无缝的跨设备体验。
    \item \textbf{响应式设计:} 界面能够自适应不同尺寸的屏幕,从手机到桌面显示器都能获得良好的视觉效果。
\end{itemize}

\subsection{主要界面设计}
\begin{itemize}
    \item \textbf{主页/仪表盘:} 采用卡片式布局,清晰展示数据结构、操作系统等各大模块入口。可加入“上次学习”、“热门可视化”等快捷方式。
    \item \textbf{可视化工作区:}
    \begin{itemize}
        \item 采用三栏式布局(理论区、画布区、控制区),结构清晰。
        \item 动画控件采用标准图标(播放、暂停、下一步),符合用户习惯。
        \item 代码区和伪代码区支持语法高亮,提升可读性。
    \end{itemize}
    \item \textbf{文档中心:} 提供清晰的导航和搜索功能,方便用户查阅理论知识和平台使用说明。
\end{itemize}

\section{非功能性需求}
\subsection{性能需求分析}

性能是用户体验的关键决定因素,直接影响平台的可用性和用户满意度。页面加载时间是用户感知性能的第一印象。根据心理学研究,用户对响应时间的感知存在三个阈值:0.1秒内的响应被认为是即时的,1秒内的响应能保持用户思维流畅,10秒以上的响应会导致用户流失。本平台设定首屏加载时间(First Contentful Paint, FCP)的目标为3秒。具体而言,初始HTML文档的加载和解析应在500ms内完成,关键CSS和JavaScript资源的加载应在1秒内完成,首屏内容的完整渲染应在3秒内完成。这要求前端资源经过充分的压缩和优化,通过代码分割(Code Splitting)和懒加载(Lazy Loading)技术,确保只有当前需要的代码被加载,并通过CDN网络就近分发。

动画流畅度对于可视化平台至关重要。人眼对帧率的感知存在阈值效应:低于24 FPS时会明显感到卡顿,达到60 FPS时动画显得流畅自然。本平台设定动画帧率目标为60 FPS,即每帧渲染时间应控制在$1000ms / 60 \approx 16.67ms$以内。这要求可视化引擎采用高效的渲染算法,避免在主线程进行复杂计算。Flutter的Skia渲染引擎能够充分利用GPU硬件加速,通过增量渲染(Incremental Rendering)和虚拟滚动(Virtual Scrolling)等优化技术,在复杂动画场景下仍能保持稳定的帧率。

用户交互的响应速度直接影响操作的流畅感。根据人机交互的研究,用户操作的响应时间应控制在200毫秒以内,此时用户感知到的是"即时反馈"。超过1秒的延迟会打断用户的思维流程,导致体验下降。本平台要求所有用户操作(如按钮点击、输入响应、动画控制)的反馈延迟不超过200毫秒,数据库查询响应时间不超过500毫秒。通过Firebase的实时数据同步能力和本地缓存策略,即使在网络条件不佳的情况下,也能保证基本的交互响应速度。

\subsection{可用性与无障碍性}

可用性(Usability)衡量系统易学、易用和令人满意的程度。根据Nielsen的可用性启发式原则,系统应具备良好的可学习性(Learnability)、效率性(Efficiency)、可记忆性(Memorability)、容错性(Error Prevention)和满意度(Satisfaction)。

本平台特别强调易学性,新用户应无需阅读长篇文档即可上手主要功能。这通过渐进式设计(Progressive Disclosure)来实现:初次访问时,系统通过引导式教程(Guided Tour)展示核心功能,复杂功能被隐藏在二级菜单中,避免信息过载。关键交互点提供上下文敏感的工具提示(Tooltip),帮助用户理解功能用途。设用户完成首次任务的平均时间为$T_0$,经过$n$次重复后的时间为$T_n$,根据学习曲线理论:
\begin{equation}
T_n = T_0 \times n^{-b}
\end{equation}
其中$b$为学习率指数(通常在0.3-0.5之间)。良好的可用性设计应使$b$值较大,即学习曲线陡峭,用户能快速掌握系统操作。

无障碍性(Accessibility)确保系统能被最广泛的用户群体使用,包括视觉、听觉、运动或认知障碍人士。本平台遵循WCAG 2.1(Web Content Accessibility Guidelines)标准的AA级别要求。在视觉设计上,考虑到约8\%男性和0.5\%女性存在不同程度的色觉缺陷(色盲或色弱),平台确保颜色不是区分信息的唯一手段,所有通过颜色传递的信息都配有形状、文字或图案作为补充标识。文本对比度遵循WCAG标准:正常文本的对比度至少为4.5:1,大号文本(18pt以上或14pt粗体以上)至少为3:1。这可以通过对比度公式验证:
\begin{equation}
Contrast Ratio = \frac{L_1 + 0.05}{L_2 + 0.05}
\end{equation}
其中$L_1$和$L_2$分别为较亮和较暗颜色的相对亮度,范围为[0, 1]。

\subsection{可扩展性架构设计}

可扩展性(Scalability)是系统应对未来需求变化和规模增长的能力,包括功能扩展性和性能扩展性两个维度。

功能扩展性通过模块化设计实现。系统采用插件化架构(Plugin Architecture),核心可视化引擎与具体的算法内容完全解耦。每个可视化内容被封装为独立的模块,包含状态机定义、渲染逻辑和交互处理。添加新的算法可视化时,开发者只需实现标准接口(Interface)并注册到内容注册表(Content Registry),核心引擎会自动发现和加载新模块。这种设计遵循开闭原则(Open-Closed Principle):系统对扩展开放,对修改封闭。设系统包含$n$个模块,添加新模块的开发成本为$C_{new}$,在紧耦合设计中,可能需要修改$k$个现有模块,总成本为:
\begin{equation}
Cost_{coupled} = C_{new} + k \times C_{modify}
\end{equation}
而在模块化设计中,由于解耦,总成本降为:
\begin{equation}
Cost_{modular} = C_{new} + C_{register}
\end{equation}
其中$C_{register}$为注册新模块的固定成本,远小于$k \times C_{modify}$。

性能扩展性通过云架构和内容分发网络实现。Firebase的自动伸缩能力使系统能够根据负载动态调整资源,理论上可以支持无限并发。设系统当前负载为$L$,处理能力为$C$,利用率为$U = L/C$。当$U$超过预设阈值(如0.7)时,触发自动扩容,增加$\Delta C$的处理能力。通过负载均衡(Load Balancing)算法,请求被分发到多个实例,平均响应时间根据排队论可近似为:
\begin{equation}
E[T] = \frac{1}{\mu - \lambda}
\end{equation}
其中$\mu$为服务率,$\lambda$为到达率。当$\lambda$接近$\mu$时,响应时间急剧上升,因此需要保持足够的余量。
