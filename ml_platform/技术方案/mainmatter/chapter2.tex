\chapter{系统架构设计}

\section{总体架构}
本平台采用典型的前后端分离架构,借助BaaS (Backend as a Service) 服务来简化后端开发和运维,使团队能够聚焦于核心可视化功能的开发。架构设计遵循关注点分离(Separation of Concerns)原则,将表示层、业务逻辑层和数据层清晰划分,每一层都有明确的职责边界。

\begin{figure}[htbp]
    \centering
    \begin{tikzpicture}[
        box/.style={rectangle, draw, fill=blue!20, minimum width=2.8cm, minimum height=1cm, align=center, font=\small},
        cloud/.style={rectangle, draw, fill=orange!20, minimum width=2.6cm, minimum height=1cm, align=center, rounded corners, font=\small},
        storage/.style={cylinder, draw, fill=green!20, minimum width=2.4cm, minimum height=1cm, align=center, shape border rotate=90, font=\small},
        layer/.style={font=\normalsize\bfseries}
    ]
        % 前端层 - 用户层
        \node[box] (web) at (0,8) {Web前端\\(Flutter Web)};
        \node[box] (android) at (4,8) {Android应用\\(Flutter)};
        
        % CDN层 - 托管层
        \node[cloud] (cdn) at (2,6) {Firebase Hosting\\+ CDN};
        
        % 外部服务
        \node[cloud] (github) at (7.5,6) {GitHub\\Actions\\CI/CD};
        
        % BaaS层 - 服务层
        \node[cloud] (auth) at (-1,3.5) {Firebase\\Authentication};
        \node[cloud] (functions) at (2,3.5) {Cloud\\Functions};
        \node[cloud] (storage) at (5,3.5) {Firebase\\Storage};
        
        % 数据层
        \node[storage] (firestore) at (2,-0.5) {Firestore\\Database};
        
        % 连接线 - 用户层到托管层
        \draw[->, thick] (web) -- (cdn);
        \draw[->, thick] (android) -- (cdn);
        
        % 连接线 - 托管层到服务层
        \draw[->, thick] (cdn) -- (auth);
        \draw[->, thick] (cdn) -- (functions);
        \draw[->, thick] (cdn) -- (storage);
        
        % 连接线 - 服务层到数据层
        \draw[->, thick] (auth) -- (firestore);
        \draw[->, thick] (functions) -- (firestore);
        \draw[->, thick] (storage) -- (firestore);
        
        % CI/CD连接
        \draw[->, thick, dashed] (github) -- node[above, font=\scriptsize] {自动部署} (cdn);
        
        % 层级标注
        \node[layer, anchor=east] at (-2.5,8) {用户层};
        \node[layer, anchor=east] at (-2.5,6) {托管层};
        \node[layer, anchor=east] at (-2.5,3.5) {服务层};
        \node[layer, anchor=east] at (-2.5,-0.5) {数据层};
    \end{tikzpicture}
    \caption{系统整体架构图}
    \label{fig:system_architecture}
\end{figure}

如图\ref{fig:system_architecture}所示,系统采用分层架构设计。前端层基于Flutter框架构建,实现所有UI展示、用户交互和可视化动画逻辑。通过单一代码库,编译生成Web应用和Android原生应用,保证了跨平台体验的一致性和开发效率。Flutter的Skia图形引擎为复杂的动画渲染提供了出色的性能保障。

托管层通过Firebase Hosting提供全球CDN加速,静态资源(HTML、CSS、JavaScript、图片)被缓存到遍布全球的边缘节点,用户请求被路由到最近的节点,极大降低了延迟。设用户距离源服务器的RTT为$T_{origin}$,距离最近CDN节点的RTT为$T_{cdn}$,则访问延迟的改善比例为:
\begin{equation}
Improvement = \frac{T_{origin} - T_{cdn}}{T_{origin}} \times 100\%
\end{equation}
典型情况下,CDN能够将延迟降低50-80\%。

服务层完全依托Google Firebase生态系统,提供用户认证、云函数、文件存储等服务。Firebase Authentication支持多种认证方式,并内置安全机制;Cloud Functions作为无服务器计算平台,可以执行后端逻辑(如数据聚合、定时任务)而无需维护服务器;Firebase Storage提供安全的文件存储,适合存储用户头像、可视化资源等媒体文件。这种BaaS架构使得前端开发者可以专注于核心业务逻辑,大幅降低了运维复杂度和成本。

数据层采用Firestore作为核心数据库,这是一个实时NoSQL文档数据库,支持离线数据持久化和跨设备实时同步。当用户在Web端更新学习进度时,数据变更会实时同步到移动端,这种无缝体验是传统架构难以实现的。

% \begin{figure}[htbp]
%     \centering
%     \includegraphics[width=0.8\textwidth]{figures/architecture.png}
%     \caption{系统总体架构图}
%     \label{fig:architecture}
% \end{figure}

\section{技术选型与论证}

技术选型是项目成功的基石,不当的选择可能导致开发效率低下、性能瓶颈或维护困难。本节将系统性地阐述各关键技术的选型理由,并通过对比分析来论证决策的合理性。

\subsection{前端框架选型}

在前端框架的选择上,我们面临着React、Vue.js、Angular和Flutter等多个选项。经过深入评估,我们最终选择Flutter 3.10+作为核心前端框架。这一决策基于以下关键考量:首先,跨平台能力是本项目的核心需求之一。Flutter允许开发者使用单一代码库同时生成Web应用和Android原生应用,这种"一次编写,到处运行"(Write Once, Run Anywhere)的能力能够将开发和维护成本降低约60\%\cite{Le2020CrossPlatform}。其次,可视化场景对渲染性能有极高要求。Flutter采用的Skia图形引擎能够直接调用GPU进行硬件加速渲染,保证复杂动画在60 FPS甚至120 FPS的流畅度,这对于展示算法的动态执行过程至关重要。此外,Flutter提供了丰富的Material Design和Cupertino风格UI组件,支持高度自定义,能够快速构建美观且符合现代审美的界面。最后,Flutter拥有活跃的开发者社区和成熟的生态系统,这为项目的长期维护和功能扩展提供了保障。

\subsection{后端服务选型}

在后端架构方面,我们采用了Backend as a Service (BaaS)模式,具体选择Google Firebase作为核心后端服务提供商。这一选择源于对项目特征的深刻理解:本项目的核心价值在于前端的可视化体验,后端主要承担用户认证、数据持久化和静态资源托管等辅助功能。Firebase提供的一体化解决方案完美契合这一需求特征。Firebase Authentication支持多种认证方式(邮箱/密码、OAuth 2.0等),并内置安全机制;Firestore作为实时NoSQL数据库,能够实现用户进度的跨设备实时同步,这对于提升用户体验至关重要;Firebase Hosting通过全球CDN网络提供静态资源托管,能够确保用户无论身处何地都能获得低延迟的访问体验;Firebase的无服务器架构采用按需计费模式,能够自动进行弹性伸缩,这大幅降低了初期运维成本和技术门槛\cite{Laguna-Salas2018Firebase}。

\subsection{技术选型对比矩阵}

为了更直观地展示技术选型的决策过程,表\ref{tab:tech_comparison}提供了详细的对比分析。

\begin{table}[htbp]
    \centering
    \caption{前端框架技术选型对比分析}
    \label{tab:tech_comparison}
    \begin{tabular}{|l|c|c|c|c|}
        \hline
        \textbf{评估维度} & \textbf{Flutter} & \textbf{React} & \textbf{Vue.js} & \textbf{Angular} \\
        \hline
        跨平台能力 & 优秀 & 中等 & 中等 & 中等 \\
        (Web+移动端) & (原生支持) & (需React Native) & (需Weex) & (需Ionic) \\
        \hline
        渲染性能 & 优秀 & 良好 & 良好 & 良好 \\
        & (60+ FPS) & (虚拟DOM) & (虚拟DOM) & (变更检测) \\
        \hline
        动画能力 & 优秀 & 中等 & 中等 & 中等 \\
        & (Skia引擎) & (CSS+JS) & (CSS+JS) & (CSS+JS) \\
        \hline
        学习曲线 & 中等 & 平缓 & 平缓 & 陡峭 \\
        & (Dart语言) & (JSX) & (模板语法) & (TypeScript) \\
        \hline
        生态成熟度 & 良好 & 优秀 & 良好 & 良好 \\
        & (快速增长) & (最成熟) & (成熟) & (企业级) \\
        \hline
        开发效率 & 优秀 & 良好 & 优秀 & 中等 \\
        & (热重载) & (快速刷新) & (热重载) & (AOT编译) \\
        \hline
        综合评分 & 9.2/10 & 8.0/10 & 7.8/10 & 7.5/10 \\
        \hline
    \end{tabular}
\end{table}

\begin{table}[htbp]
    \centering
    \caption{后端服务选型对比分析}
    \label{tab:backend_comparison}
    \begin{tabular}{|l|c|c|c|}
        \hline
        \textbf{评估维度} & \textbf{Firebase} & \textbf{自建后端} & \textbf{AWS Amplify} \\
        \hline
        初期成本 & 低 & 高 & 中等 \\
        & (按需计费) & (服务器+人力) & (按需计费) \\
        \hline
        开发速度 & 快 & 慢 & 中等 \\
        & (开箱即用) & (从零搭建) & (配置复杂) \\
        \hline
        可扩展性 & 优秀 & 优秀 & 优秀 \\
        & (自动伸缩) & (手动配置) & (自动伸缩) \\
        \hline
        维护难度 & 低 & 高 & 中等 \\
        & (全托管) & (需专人) & (部分托管) \\
        \hline
        实时同步 & 原生支持 & 需自实现 & 支持 \\
        & (Firestore) & (WebSocket) & (AppSync) \\
        \hline
        国内访问 & 受限 & 无限制 & 受限 \\
        & (需代理) & (自主可控) & (需代理) \\
        \hline
        综合评分 & 8.5/10 & 7.0/10 & 7.8/10 \\
        \hline
    \end{tabular}
\end{table}

从表\ref{tab:tech_comparison}和表\ref{tab:backend_comparison}可以看出,在本项目的特定需求场景下,Flutter和Firebase的组合能够提供最优的性价比和开发效率。

\section{部署架构与CI/CD流程}
\subsection{部署策略}
\begin{itemize}
    \item \textbf{Web应用:} 通过Firebase Hosting进行部署。开发者只需将Flutter Web编译生成的静态文件(HTML, CSS, JS)上传,即可通过全球CDN网络为用户提供低延迟的访问体验。Firebase Hosting支持自定义域名、SSL证书自动配置和版本回滚功能。
    
    \item \textbf{Android应用:} 通过GitHub Releases功能发布预编译的APK安装包,用户可直接下载安装。APK文件包含完整的应用代码和资源,无需额外依赖。未来可考虑上架Google Play或国内主流安卓应用商店(如华为应用市场、小米应用商店),以提升应用的可发现性和用户信任度。
    
    \item \textbf{版本管理:} 采用语义化版本号(Semantic Versioning),格式为MAJOR.MINOR.PATCH(如1.0.0),明确区分重大更新、功能新增和bug修复。
\end{itemize}

\subsection{持续集成/持续部署 (CI/CD)}
为实现开发流程的自动化和标准化,项目将配置完整的CI/CD流水线:
\begin{itemize}
    \item \textbf{工具链:} 采用GitHub Actions作为CI/CD平台,利用其与GitHub深度集成的优势。
    
    \item \textbf{CI流程:} 当代码推送到主分支或提交Pull Request时,自动触发以下步骤:
    \begin{enumerate}
        \item 代码静态分析(Lint检查)
        \item 单元测试和集成测试执行
        \item Flutter应用编译(Web和Android)
        \item 构建产物(APK、Web静态文件)的生成
    \end{enumerate}
    
    \item \textbf{CD流程:} 当代码合并到主分支且所有测试通过后:
    \begin{enumerate}
        \item Web版本自动部署到Firebase Hosting
        \item Android APK自动上传到GitHub Releases
        \item 自动生成版本变更日志(Changelog)
        \item 发送部署成功通知
    \end{enumerate}
    
    \item \textbf{质量保障:} 在部署前自动运行测试套件,确保代码质量;支持手动触发回滚操作,快速恢复到上一个稳定版本。
\end{itemize}
