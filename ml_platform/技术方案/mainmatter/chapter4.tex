\chapter{数据库设计}

数据库设计是应用开发的关键环节,直接影响系统的性能、可扩展性和维护性。本项目采用Google Firestore作为核心数据存储方案,这是一个完全托管的NoSQL文档数据库,提供了实时同步、离线支持和自动扩展等特性。

\section{数据模型设计理念}

Firestore采用集合-文档(Collection-Document)的层次结构,这是一种灵活且直观的数据组织方式。与传统的关系型数据库(RDBMS)不同,NoSQL文档数据库不强制要求固定的表结构(Schema),每个文档可以包含不同的字段集合,这种灵活性特别适合快速迭代和需求变更频繁的项目场景。

在Firestore中,集合(Collection)类似于关系型数据库中的表(Table),但不需要预先定义结构。文档(Document)是数据的基本单位,以JSON格式存储,每个文档由键值对组成,值可以是基本类型(字符串、数字、布尔值)、复杂类型(数组、嵌套对象)或特殊类型(时间戳、地理位置)。文档之间通过文档ID唯一标识,文档ID可以自动生成或手动指定。

Firestore与关系型数据库的性能特征存在显著差异。关系型数据库擅长复杂查询和多表关联(Join),但在高并发读写场景下可能成为瓶颈。Firestore采用了不同的设计理念:通过数据冗余(Data Redundancy)和反规范化(Denormalization)来换取查询性能。设查询的平均响应时间为$T_q$,在关系型数据库中,涉及$k$张表的关联查询的时间复杂度约为:
\begin{equation}
T_q^{RDBMS} = O(n_1 \times n_2 \times ... \times n_k)
\end{equation}
而在Firestore中,通过预先将相关数据嵌入到单个文档中,查询时间降为:
\begin{equation}
T_q^{Firestore} = O(1)
\end{equation}
这种设计trade-off了存储空间和写入一致性维护的复杂度,换取了极致的读取性能。

表\ref{tab:db_comparison}对比了Firestore与传统关系型数据库的特征差异。

\begin{table}[htbp]
    \centering
    \caption{Firestore与关系型数据库对比}
    \label{tab:db_comparison}
    \begin{tabular}{|l|p{5.5cm}|p{5.5cm}|}
        \hline
        \textbf{特性} & \textbf{Firestore (NoSQL)} & \textbf{关系型数据库 (SQL)} \\
        \hline
        数据模型 & 文档-集合层次结构,JSON格式,灵活Schema & 表-行-列结构,固定Schema \\
        \hline
        查询能力 & 简单查询高效,复杂关联困难 & 支持复杂SQL查询和多表Join \\
        \hline
        扩展性 & 水平扩展,自动分片 & 垂直扩展为主,水平扩展复杂 \\
        \hline
        事务支持 & 支持单文档和小批量事务 & 完整ACID事务支持 \\
        \hline
        读性能 & 极高(通过冗余) & 中等(需Join) \\
        \hline
        写性能 & 高(需维护冗余) & 中等 \\
        \hline
        数据一致性 & 最终一致性 & 强一致性 \\
        \hline
        实时同步 & 原生支持 & 需额外实现 \\
        \hline
        离线支持 & SDK内置 & 需自行实现 \\
        \hline
        学习曲线 & 平缓 & 陡峭(需学习SQL) \\
        \hline
    \end{tabular}
\end{table}

\section{核心集合设计}
\subsection{users 集合}
\begin{itemize}
    \item 文档ID: \texttt{uid} (来自Firebase Auth)
    \item 文档内容:
\begin{verbatim}
{
 "displayName": "用户名",
 "email": "user@example.com",
 "photoURL": "头像URL",
 "createdAt": "注册时间戳",
 "progress": {
   "data_structures": {
     "quick_sort": "completed",
     "dijkstra": "in_progress"
   },
   "operating_systems": { ... }
 }
}
\end{verbatim}
\end{itemize}

\subsection{visualizations 集合}
\begin{itemize}
    \item 文档ID: \texttt{visualization\_id} (如 \texttt{quick\_sort})
    \item 文档内容:
\begin{verbatim}
{
 "title": "快速排序",
 "description": "快速排序算法的详细介绍...",
 "category": "data_structures",
 "tags": ["排序", "分治"],
 "difficulty": "medium",
 "version": "1.0"
}
\end{verbatim}
\end{itemize}

\subsection{user\_experiments 集合}
\begin{itemize}
    \item 文档ID: 自动生成
    \item 文档内容:
\begin{verbatim}
{
 "uid": "所属用户uid",
 "visualization_id": "quick_sort",
 "title": "我的快速排序实验",
 "input_data": "[10, 5, 2, 7, 6, 1]",
 "savedAt": "保存时间戳"
}
\end{verbatim}
\end{itemize}
