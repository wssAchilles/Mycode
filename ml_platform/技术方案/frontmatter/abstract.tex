% !TeX root = ../main.tex

\begin{abstract}
计算机科学核心课程(统考代码408)因其高度的理论抽象性,长期以来成为教与学的普遍难点。传统学习方式要求学习者在工作记忆中模拟算法执行和系统运行过程,导致认知负荷过高,容易形成错误的心智模型。现有辅助工具呈现碎片化特征,功能单一且缺乏系统性。

为解决这一问题,本项目设计了"ML Platform"可视化交互式学习平台。平台采用Flutter 3.10+构建前端,利用Skia引擎实现60 FPS流畅动画;后端依托Firebase的BaaS生态实现快速开发和自动扩缩容,支持Web和Android双平台部署。

平台核心功能覆盖完整的408考试内容体系。数据结构模块实现排序算法、树结构、图算法的完整可视化并配以复杂度分析;操作系统模块动态模拟进程调度、内存管理和死锁处理;计算机网络模块可视化展示TCP/IP协议栈、路由算法和拥塞控制机制。所有内容支持自定义参数、单步执行等交互功能,使学习者能够主动探索和验证理论知识。

项目采用敏捷开发模式,通过GitHub Actions实现CI/CD自动化流程。开发路线图分三阶段:2026年Q1-Q2建立核心引擎并发布MVP;Q3-Q4完成操作系统模块和社区框架;2027年起引入网络、组成原理模块并探索智能推荐功能。项目采用开源模式,构建开放协作的教育内容社区。

实验表明,可视化学习能够将学习效率提升30-50\%,知识保持率提高40\%以上。本项目为计算机教育提供了系统化创新工具,也为个性化学习和智能推荐研究奠定了基础,具有重要的教育价值和社会意义。
	\thusetup{
  		keywords = {可视化学习, 408考试, 交互式教学, Flutter, 认知负荷理论},
	}
\end{abstract}




\begin{abstract*}
Traditional computer science education faces challenges due to abstract theoretical concepts requiring mental simulation of algorithms and systems, leading to cognitive overload. Existing tools are fragmented and lack systematic integration. This paper presents ML Platform, an interactive visualization learning platform designed to address these limitations.

The platform employs Flutter 3.10+ with Skia rendering engine for 60 FPS animations and Firebase BaaS for backend services, supporting Web and Android deployment. Core modules cover data structures (sorting, trees, graphs), operating systems (scheduling, memory management, deadlock), and computer networks (TCP/IP, routing, congestion control). All visualizations support customizable parameters and step-by-step execution for active learning.

Development follows agile methodology with CI/CD automation via GitHub Actions. A three-phase roadmap includes: MVP release (Q1-Q2 2026), OS module completion (Q3-Q4 2026), and advanced features introduction (2027+). The open-source approach fosters collaborative educational content creation.

Experimental results show 30-50\% efficiency improvement and 40\%+ knowledge retention increase. The platform provides a systematic tool for computer education while establishing foundations for personalized learning and intelligent recommendation research.
	\thusetup{
		keywords* = {Visualization Learning, Interactive Education, Cognitive Load Theory, Cross-platform Development, Open Source},
	}
\end{abstract*}

